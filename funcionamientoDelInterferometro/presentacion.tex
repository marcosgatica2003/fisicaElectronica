\documentclass{beamer}
\usetheme{default}

\title{FUNCIONAMIENTO DE UN INTERFERÓMETRO}
\author{por Marcos Raúl Gatica}
\date{}

\begin{document}
\begin{frame}[plain]
    \maketitle
\end{frame}
\begin{frame}
    \frametitle{Principio básico de funcionamiento}
    \textbf{
     El funcionamiento del interferómetro se basa en la superposición de ondas, generalmente de la luz.
    }
    \newline
    \textbf{
    Cuando dos o más ondas interfieren, el resultado es una combinación de las mismas.
    }
\end{frame}
\end{document}
